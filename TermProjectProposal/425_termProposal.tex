% personal latex template

% vegito2002@gmail.com
\documentclass{article}
\usepackage{times}
\usepackage{mathptmx}
\usepackage{multirow}
\usepackage{enumitem}
\usepackage[letterpaper, margin=1in]{geometry}
\usepackage{hyperref}
\usepackage{amsmath}
\usepackage{mathtools}
\DeclarePairedDelimiter{\ceil}{\lceil}{\rceil}
\DeclarePairedDelimiter{\floor}{\lfloor}{\rfloor}
\usepackage{graphicx}
\usepackage{listings}
\usepackage{lipsum}
\usepackage{courier}
% \usepackage{tabularx}
\usepackage[export]{adjustbox}
% \usepackage[landscape]{pdfpages}
\usepackage{enumitem}
\usepackage{spreadtab}
% \usepackage{enumerate}


\graphicspath{ {images/} }

\usepackage[dvipsnames]{xcolor}  %% Allow color names
\lstset
{ %Formatting for code in appendix
  belowcaptionskip=1\baselineskip,
  xleftmargin=\parindent,
  language=Java,   %% Change this to whatever you write in
  breaklines=true, %% Wrap long lines
  basicstyle=\ttfamily,
  % numbers=left,
  % stepnumber=1,
  commentstyle=\itshape\color{Gray},
  stringstyle=\color{Orange},
  keywordstyle=\bfseries\color{WildStrawberry},
  identifierstyle=\color{Blue},
}
\usepackage{titlesec}
\usepackage{amssymb}
\newenvironment{claim}[1]{\par\noindent\underline{Claim:}\space#1}{}
\newenvironment{claimproof}[1]{\par\noindent\underline{Proof:}\space#1}{\leavevmode\unskip\penalty9999 \hbox{}\nobreak\hfill\quad\hbox{$\blacksquare$}}

\setcounter{secnumdepth}{4}

\titleformat{\paragraph}
{\normalfont\normalsize\bfseries}{\theparagraph}{1em}{}
\titlespacing*{\paragraph}
{0pt}{3.25ex plus 1ex minus .2ex}{1.5ex plus .2ex}

\begin{document}

% Article top matter
\title{\textsc{EN 600.425 Declarative Methods \\ - Term Project Proposal}\\ \textbf{\textsc{Better } \texttt{diff}}} %\LaTeX is a macro for printing the Latex logo
\author{Guoye Zhang, Qiang Zhang\\
        % \texttt{Hopkins ID: 92CC3E}
        }  %\texttt formats the text to a typewriter style font
\date{\today}  %\today is replaced with the current date
\maketitle

\section{Introduction}
\texttt{diff} is a commonly used utility. It has been widely adopted in spreading circumstances of file comparison and patch generation. Furthermore, \texttt{diff} is also the basis of version control systems, including \texttt{git}.\\

What \texttt{diff} solves is essentially the \textit{longest common subsequence} problem. The algorithm \texttt{diff} employs is dynamic programming in spirit, targeted at finding least number of line changes that can lead from the old version file to the newer version. Usually, \texttt{diff}, does a pretty good job  in solving this problem and deserves all its compliments. But from time to time, \texttt{diff} does not always produce the perfect, or the more intuitively desirable solution. For lack of a better example, here is a case where \texttt{diff} only manages to find suboptimal solution:\\
\begin{lstlisting}
for(int i=0; i<10; i++) {
    System.out.println("First line");
}
for(int i=0; i<10; i++) {
    System.out.println("Second line");
}
for(int i=0; i<10; i++) {
    System.out.println("Third line");
}
\end{lstlisting}

And when we try to delete the second block entirely, what we expect would be:
\begin{lstlisting}
  for(int i=0; i<10; i++) {
      System.out.println("First line");
  }
 -for(int i=0; i<10; i++) {
 -    System.out.println("Second line");
 -}
  for(int i=0; i<10; i++) {
      System.out.println("Third line");
  }
\end{lstlisting}

\pagebreak
But what we have instead is actually:
\begin{lstlisting}
  for(int i=0; i<10; i++) {
      System.out.println("First line");
  }
  for(int i=0; i<10; i++) {
 -    System.out.println("Second line");
 -}
 -for(int i=0; i<10; i++) {
      System.out.println("Third line");
  }
\end{lstlisting}

The above solutions are equivalent in effect, but can be confusing regarding the purpose of \texttt{diff} editing history in the first place. This is only a primitive example that we managed to find shortly that is reproducible. In real world applications, a similar malfunctioning patching that is a larger scale can be considerably frustrating and counterproductive.\\

We intend to devise a new \texttt{diff} strategy (including algorithms and implementation) that takes scoping issues into consideration. Also dynamic programming is not the best choice of solve in our opinion since dynamic programming's search heuristic is usually exhaustive and ineffeicient. We intend to change the problem into a MAX-SAT problem for which numerous smart heuristics have been researched and implemented.\\

We also intend to extend the utility's functionality so that it can also \texttt{diff} moving and copying.\\

The basic guideline of our design would be adding an intermediate layer between the original text file and the modeling of \texttt{diff}. We want our algorithm to be able to produce a solution of line changes that does not break the block/scoping structure of the original file. Our utility would first recognize all the blocks, subject to different definitions regarding different kinds of text, in the file. We apply a hashing to each block and use the hash value as the basis of changes calculation. Using hashings like locality-sensitive hashing, our solver would try to match a block in the old version to a block in the newer version with a probability that is highly positively associated with the similarity between the two blocks hashing values. By this heuristic, we will be more likely to find a line changes solution that preserves the file's block structure as much as possible.\\

Now that we are able to locate the newer version of some block of the older version file, we have to look into the file and find the editing changes within the block on a line-by-line basis. There are currently many heuristics for this problem, including the one the current \texttt{diff} is using. Although we are not final on the choice of this, MAX-SAT seems very auspicious as far as we are concerned now.

\section{Dealing with Code Files}
Code files are relatively easier to process amongst all sorts of text files because block structure in code files are usually well defined and even well labeled. We need not calculate the hash for each block in this situation. We can just use some technique to extract the block name (like method name in Java) as the key for block matching. \\

One subtle issue would arise when overloading comes into place. There will be a situation where the user just changed a 2-arg method into a 3-arg method. There will also be a situation where a user did not change the 2-arg method, but rather just added a 3-arg one. Our algorithm should be able to distinguish between such situations. One plan would be to match blocks by block name first. Then compare the number of blocks of the same name in each file. If The numbers are equal, then we directly match the two blocks, without further differentiating more information about the two blocks. But if the number of blocks under the same name are not equal in the two files, then additional information are needed for matching the two files. We further take into consideraion of the entire signature/header line of the block. We treat the block in the new version file that has the same signature line as the modified version of the same block in the orginal file. The other block in the new version file that has the same name but different signature line shall be treated as an lines addition.\\

We realize that there will be other issues that may come up and break our algorithm. Thus as tempting as it is to skip hashing the block body lines altogether for code files, we might still end up implementing some minimal version of hashing for the block body lines. 

\section{Dealing with Non-Code Files}
So-called non-code files are just regular files. Could be an article, or letter. The tricky part in such situations is to define and implement an effeicient but reasonable way to deliminating different blocks in the file. Natural block delimination based on paragraphs seems reasonable, but sometimes paragraph information may not be available. We are still contemplating about the right way to \texttt{block-ize} the file in this situation. \\

Once proper block structuring scheme is established, the problem remaining is devising a working hashing for such files.
Based on the pre-mentioned LSH algorithms, we hope to develop a strategy to hash blocks so that blocks in different files are more likely to be matched together when their hash values are close.

Two blocks in two files having equal hash values should be interpreted as mere copying or moving. 

\section{Summary}
Our primary goal is to improve \texttt{diff} in such a way that the algorithm no longer tries to match lines in differnt files on a basis of line positions (line numbers), but rather on their aggregate attribute, namely, on lines' relative relationship between each other. By doing so, our version of \texttt{diff} would also be able to recognize editing changes like copying and moving, which is transparent to current implementation of \texttt{diff}.

We are advised by Prof. Eisner that our algorithm should be implemented in a tool that can trace text migration and inheritance between different files. The tool should be able to display these changes/differences and the inheritance relationship in a friendly GUI. Our plan for such a tool's design is not ripe yet, but the first intuition is to other than showing the line difference results (including moving and copying) between the two different files, we also add some additional visuals to be able to display the temporal relationships between files. 

A naive idea would be to directly use the file mofication timestamp as the basis of judging the temporal relationships between two blocks in two files that are recognized as matching.
The problem with this idea is, block 1 in file 1 might be modified earlier than block 2 in file 2, but if we tried to modify some part of file 1 other than block 1 later after we finished editing block 2 of file 2, the file modification time would not be a solid basis for judging the temporal precedence between the two blocks.

Our current thought is to create an additional shadow file for each file. And our tool would possibly be in the form of an editor. Now, we do not compute the hash values for the blocks of a file only when \texttt{diff} is run on it. Rather, we hash the blocks of each file every time the file is saved. The hash values are stored in the shadow file, together with current \texttt{save}'s timestamp. This shadow file contains only around two words for each block of a file, making it inexpensive to maintain. Each time we have to run \texttt{diff} on two files, we not only display the line differences between the files using the LSH hashing as described before, we also index into each of the two file's shadow log. We use the current \texttt{diff}'s calculated hash values as index to find the most matching timestamp in the shadow logs, and integrate such information into the display of the GUI. Of course this is only a sketchy scheme for now, and hopefully we will come up with something more compact and effective later in the semester.

% \begin{thebibliography}{9}
% %The \bibitem is to start a new reference.  Ensure that the cite_key is
% %unique.  You don't need to put each element on a new line, but I did
% %simply for readability.
%     \bibitem{RFC2616}
%       RFC 2616\\% \href{http://www.texample.net/tikz/resources/}{Tex Example Site}


% \end{thebibliography} %Must end the environment


\end{document}  %End of document.